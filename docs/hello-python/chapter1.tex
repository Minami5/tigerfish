\chapter{Python for Machine Learning: Walkthrough}
	本講義は、Pythonを使った実習を行います。
	この言語は、データサイエンス・機械学習の分野で確固たる地位を獲得しているプログラミング言語であり、
	高い可読性を持つインタープリタ型言語です。
	この章ではPythonの基本的文法と
	いくつかの有名なライブラリの使い方を簡単に紹介します。
	受講者全員がC言語等でのプログラミング経験があるはずなので、
	すぐにマスター出来るでしょう。では、早速Pythonの世界に触れてみましょう。

	なお、本資料はGoogle Colaboratory\footnote[1]{\url{https://colab.research.google.com/}}での利用を想定しています。
	このサービスは無料で利用可能ですが、Googleアカウントが必要です。各自で用意して下さい。
	
	また、Pythonには2系と3系がありますが、
	本講義ではPython 3系の利用を前提としています。
	2020年にサポートが終了するPython 2系を未だに使っていると、
	小学生にけなされる\cite{python_rl}そうなので考慮するだけ時間の無駄です。やめましょう。
	
	\section{Hello, World!}	
		全てのプログラミング言語学習はこのコードから始まります。
		\texttt{Hello, World!}を出力してみましょう。 \\
		
		\texttt{print("Hello, World!")} \\
		
		C言語と違い、たった一行\texttt{print()}を呼び出すだけで出力できたと思います。
	
	\section{変数と配列}
		\subsection{変数}
			Pythonは動的型付けの言語です。
			どういう事かを変数を宣言しながら見ていきましょう。
	\begin{lstlisting}[basicstyle=\ttfamily, language=python, tabsize=4, frame=single]
a = 1	# int
b = 2.0	# float
	\end{lstlisting}
		
			コードを見て気付くかと思いますが、
			PythonではC言語と違って変数宣言時データ型を指定しません。
			これは、データ型が存在しない訳でなく、
			コードの実行時に変数の中身に応じて動的に型が決定されます。
			
		\subsection{配列}
			Pythonでは、次のように配列を宣言します。
			
	\begin{lstlisting}[basicstyle=\ttfamily, language=python, tabsize=4, frame=single]
x = [0, 1, 2, 3, 4, 5, 6, 7, 8, 9]
	\end{lstlisting}
			
			また、次節で解説する\texttt{for文}を使った内包表記と呼ばれる書き方を方法もあります。
	\begin{lstlisting}[language=python, basicstyle=\ttfamily, tabsize=4, frame=single]
x = [i for i in range(10)]
	\end{lstlisting}
	
			
			これは、\texttt{[式 for 変数 in iterableオブジェクト]}
			という書き方です。
			なお、\texttt{iterableオブジェクト}は配列や辞書などのことを指します。
			
			また、配列の操作は次のように行います。
			
	\begin{lstlisting}[language=python, basicstyle=\ttfamily, frame=single]
y = [1, 2, 3]
y.append(4)
print(y)	# [1, 2, 3, 4]
print(y[1])	# 2
z = [i for i in range(10)]
print(z[:5])	# [0, 1, 2, 3, 4]
print(z[2:5])	# [2, 3, 4]
print(z[5:])	# [5, 6, 7, 8, 9]
	\end{lstlisting}
			
	\section{ループ, 条件分岐}
		\subsection{forループ}
			PythonのforループはC言語とは趣が違います。
			基本的に、\texttt{range関数}や配列などの\texttt{iterableオブジェクト}を
			使った方法が用いられます。
			
	\begin{lstlisting}[language=python, basicstyle=\ttfamily, frame=single, tabsize=4]
for i in z:
	print(i)
	
for j in reversed(z):
	print(j)

	\end{lstlisting}
			
			ここで気付くと思いますが、
			C言語の様に\texttt{{}}を使って制御範囲を示すのではなく、
			インデントによって範囲を表します。
			このインデントは、tabまたはspace 4回分とされています。
			
			また、\texttt{range関数}には次の様に詳細に
			始点、終点、ステップを指定することができます。
	\begin{lstlisting}[language=python, basicstyle=\ttfamily, frame=single, tabsize=4]
for i in range(3, 13, 2):
	print(i)
	\end{lstlisting}
			
			
			ただし、\texttt{range関数}ではステップ数は整数型のみしか扱えません。
			実数型を扱う為には後述のNumPyの関数を使う必要があります。
			
		\subsection{whileループ}
			\texttt{while文}は次の様に書きます。
\begin{lstlisting}[language=python, basicstyle=\ttfamily, frame=single, tabsize=4]
i = 0
while i < 5:
	print(i)
	i += 1
\end{lstlisting}
			
		\subsection{条件分岐}
			Pythonで条件分岐を行う手法は\texttt{if文}のみです。
			Pythonには残念な事に\texttt{switch文}が用意されていません。
			
			\texttt{if文}は次の通りです。
\begin{lstlisting}[language=python, basicstyle=\ttfamily, frame=single, tabsize=4]
x = 100
if x % 5 == 0:
	print("A")
elif x % 3 == 0:
	print("B")
else:
	print("C")
\end{lstlisting}
			
		\subsection{FizzBuzz問題}
			ここまで理解出来たらFizzBuzz問題に挑戦していましょう。
			この問題を知らない4年生や大学院生はいないはずですが、
			念のためルールを書いておくと、
			『1から順に数字を出力し、
			3の倍数の時は数字の代わりに\url{Fizz}、
			5の倍数の時は代わりに\url{Buzz}、
			15の倍数の時は代わりに\url{FizzBuzz}と出力させる』
			という簡単な問題です。
			
			各自で解いてみて下さい。
	
	\newpage
	\section{関数, オブジェクト指向}
		\subsection{関数の宣言}
			関数の定義には\url{def}を使います。
\begin{lstlisting}[language=python, basicstyle=\ttfamily, frame=single, tabsize=4]
def f(a, b):
	return a + b
\end{lstlisting}
			
			また、デフォルト引数を設定する事も出来ます。
\begin{lstlisting}[language=python, basicstyle=\ttfamily, frame=single, tabsize=4]
def g(a, b=100):
	return a + b

print(g(3))	# Omit the argument b
print(g(3, 2))
\end{lstlisting}
			
			また、引数を明示的に指定することも出来ます。\\
			ここでは、式$h(x, y, z) = x^y + z$を関数として実装した例で見てみましょう。
\begin{lstlisting}[language=python, basicstyle=\ttfamily, frame=single, tabsize=4]
def h(x, y=2, z=3):
	return x ** y + z

print(h(2, y=3, z=4))
print(h(y=2, x=3, z=6))
\end{lstlisting}
		
		\newpage
		\subsection{オブジェクト指向}
			Pythonにおいてクラスの宣言は次のように行います。
\begin{lstlisting}[language=python, basicstyle=\ttfamily, frame=single, tabsize=4]
class Person:
	def __init__(self, name, address):
		self.__name = name
		self.__address = address
	
	@property
	def name(self):
		return self.__name

	@property
	def address(self):
		return self.__address
		
	@address.setter
	def address(self, new_address):
		if type(new_address) is str:
			self.__address = new_address
			
	def say(self, text):
		print(text)

kaguya = Person("Kaguya Shinomiya", "Kyoto")
print("My name is {}".format(kaguya.name))
print("I lived in {}.".format(kaguya.address))

kaguya.address = "Sengakuji"
print("I'm living in {} now".format(kaguya.address))

kaguya.say("How cute...")
\end{lstlisting}
			
			ここで、\texttt{\_\_init\_\_}はイニシャライザ
			(もしくはコンストラクタ)と呼ばれ、クラスがインスタンス化される時に
			一番最初に呼ばれ処理されます。
			また、\texttt{@property}が頭についている関数は\texttt{getter}、
			\texttt{@address.setter}が\texttt{addressプロパティ}の\texttt{setter}で、
			それぞれ\texttt{getter}や\texttt{setter}を利用したい場合のみ記述すれば良いので必須ではありません。
			また、\texttt{self.\_\_name}などの変数の先頭に\texttt{\_\_}がある変数は、いわゆる\texttt{private変数}です。
			ただ、Pythonの仕様上、擬似的なもので完全な\texttt{private}ではありません。
			
			\texttt{getter/setter}について分からなければ各自で調べてください。

	\section{NumPy, Matplotlib}
		\subsection{NumPy}
			NumPyはPythonで数値計算を効率的に行うためのライブラリです。
			このライブラリがあったお陰で、
			Pythonが今の様な数値計算・データサイエンス・機械学習で
			確固たる地位を確立出来たと言っても過言ではないでしょう。
			さっそく、NumPyを触ってみましょう。
			
			最も単純な配列の宣言方法は次の通りです。
			
\begin{lstlisting}[language=python, basicstyle=\ttfamily, tabsize=4, frame=single]
import numpy as np
a = np.array([2, 3, 5, 7, 8])
print(a.dtype)
\end{lstlisting}

			次に、新たに配列を宣言し配列の形状やデータ型を確認してみましょう。

\begin{lstlisting}[language=python, basicstyle=\ttfamily, tabsize=4, frame=single]
x = np.array([i for i in np.arange(15.)]).reshape(3, 5)
print(x.shape)	# 配列の形状
print(x.ndim)	# 配列の次元数
print(x.size)	# 配列の要素数
print(x.T)		# 転置配列
print(x.dtype)	# 配列のデータ型
\end{lstlisting}
			
			\newpage
			次に配列の基本的な計算は次の様に行います。
\begin{lstlisting}[language=python, basicstyle=\ttfamily, tabsize=4, frame=single]
print(x.sum())	# 合計
print(x.mean())	# 平均
print(x.max())	# 最大値
print(x.min())	# 最小値
\end{lstlisting}

			ちなみに、多次元配列の場合、次の様にする事で軸ごとの計算を行います。
			
\begin{lstlisting}[language=python, basicstyle=\ttfamily, tabsize=4, frame=single]
print(x.sum(axis=0))	# 列ごとの合計
print(x.sum(axis=1))	# 行ごとの合計
\end{lstlisting}
		
			ここで紹介した機能はPythonのごく一部です。詳しくは公式ドキュメント\footnote{http://www.numpy.org/} 等を参照して下さい。
		
		\subsection{Matplotlib}
			Matplotlibの基本的な使い方を紹介します。
			なお、このライブラリはNumPyと同じく非常に多機能なので、
			詳しくは公式ドキュメント\footnote{https://matplotlib.org/} 等を参照して下さい。
			
\begin{lstlisting}[language=python, basicstyle=\ttfamily, frame=single,tabsize=4]
import numpy as np
import matplotlib.pyplot as plt

a = np.linspace(-5, 5)
b = np.arange(len(a))
plt.plot(a, b)
plt.show()
\end{lstlisting}
			